\documentclass{report}
\usepackage{MCC}

\def\footauthor{Thomas COUCHOUD \& Victor COLEAU}
\title{Système multimédia - TP6}
\author{Thomas COUCHOUD\\\texttt{thomas.couchoud@etu.univ-tours.fr}\\Victor COLEAU\\\texttt{victor.coleau@etu.univ-tours.fr}}

\rowcolors{1}{white}{white}
\begin{document}
	\mccTitle

	\chapter{Envoie d’une vidéo via le protocole UDP}
		\section{Serveur}
			L'objectif de cette question est de réaliser un programme python permettant de filmer et de streamer en live sur 3 adresses différentes.
	
			On commence par initialiser un socket avec les paramètres AF\_INET (utiliser Internet) et SOCK\_DGRAM (protocole UDP).
			Ensuite on créé un objet PiCamera est initialisons les paramètres suivants :
			\begin{easylist}[itemize]
				@ \textbf{vflip} : inversion verticale de l'image.
				@ \textbf{hflip} : inversion horizontale de l'image.
				@ \textbf{resolution} : résolution de l'image. Ici à 640x480 pixels.
				@ \textbf{framefate} : framerate théorique de capture. Ici à 32 images par seconde.
			\end{easylist} 
	
			Un objet rawCapture permettra par la suite d'obtenir tous les pixels de l'image enregistrée.
	
			Une boucle For itérant sur la méthode camera.capture\_continuous() permet de traiter chaque image tant que le flux vidéo est actif.
	
			Pour chaque image, on commence par la passer en nuance de gris grâce à la méthode cv2.cvtColor() puis on la redimensionne grâce à la méthode cv2.resize().
	
			Enfin, grâce à la méthode sock.sendto() appelée 3 fois, nous pouvons envoyer sur les 3 streams l'image précédemment traitée. Dans notre cas, les sockets transfèrent les données vers notre client, soit la VM utilisée, sur les ports 5006, 5007 et 5008.

		\section{Client}
			Pour le client, l'objectif est d'exposer un des 3 ports précédents afin de recevoir et afficher les données du stream.
	
			On passe en paramètre le port souhaité et le stockons dans une variable.
	
			On ouvre un socket similaire à celui du serveur et l'assignons à l'ip et au port voulus.
	
			Ensuite, tant que le flux vidéo continu, on :
			\begin{easylist}[itemize]
				@ Reçoit des octets grâce à la méthode sock.recvfrom(). La taille lue correspond à la taille d'une image.
				@ On créé une image à partir de ces données grâce à la méthode np.frombuffer().
				@ On redimensionne l'image en 320x180 pixels avec la méthode np.reshape().
				@ Enfin, on affiche l'image obtenu grâce à la méthode cv2.imshow().
			\end{easylist}
	
			Par soucis de commodité, nous avons ajouté une condition d'arrêt si l'on appuie sur la touche "q".
	
		\section{Serveur 2}
			Dans cette partie, plusieurs améliorations ont été ajoutées au serveur.
	
			\begin{easylist}[itemize]
				@ Chaque stream est envoyé sur un port différent. Ici les ports 5006, 5007 et 5008. Pour ce faire, chaque stream est associé à un thread grâce à une classe \textit{videoprocess} dérivée de \textit{thread}.
				@ Le nombre d'images par seconde peut être choisit au lancement (premier paramètre du constructeur de l'objet \textit{videoprocess}.
				@ Pour afficher un texte sur l'image, la méthode setImage de l'objet \textit{videoprocess} appelle cv2.putText().
			\end{easylist}
	
			Après tests, pour 1 client, nous arrivons à environ 13 images par seconde.
			
			Cependant, si l'on lance 3 clients simultanément, nous pouvons atteindre jusqu'à 10 images par seconde sur chacun d'entre eux.
			
		\section{Serveur 3}
			Dans cette version nous avons implémenté plusieurs changements:
			\begin{easylist}[itemize]
				@ Max FPS de la caméra à 60
				@ Capture de la caméra directement à la résolution finale, permettant ainsi d'éviter les redimentionnements.
			\end{easylist}
		
		\section{Screenshot}
			\img{C2.png}{}{}
		
\end{document}
