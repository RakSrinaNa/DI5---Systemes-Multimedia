\documentclass{report}
\usepackage{MCC}

\def\footauthor{Thomas COUCHOUD \& Victor COLEAU}
\title{Système multimédia - TP5}
\author{Thomas COUCHOUD\\\texttt{thomas.couchoud@etu.univ-tours.fr}\\Victor COLEAU\\\texttt{victor.coleau@etu.univ-tours.fr}}

\rowcolors{1}{white}{white}
\begin{document}
	\mccTitle

	\section{Prédiction aléatoire}
		Avec la prédiction aléatoire, nous obtenons les RMSE suivants:
		
		\begin{tabularx}{\textwidth}{|X|X|X|X|X|X|}
			\hline
			Valeur de prédiction & 1 & 2 & 3 & 4 & 5\\\hline
			RMSE & 2.77 & 1.90 & 1.24 & 1.22 & 1.85\\\hline
		\end{tabularx}
		
		La moyenne de ces valeurs est de 1.8, ce qui semble impliquer que les notes données ne sont pas équitablement réparties entre 1 et 5.
		D'après le tableau ci-dessus, l'erreur la plus petite s'obtient avec un prédicateur égal à 3 ou 4, on peut en déduire que la majorité des notes se situe aux autour de 3 et 4.
		Si l'on fait la moyenne des notes données, on obtient 3.52, ce qui confirme notre hypothèse.
		
	\section{Prédiction basique}
		Avec la prédiction basique, nous obtenons un RMSE de 0.938.
		
		Le prédicateur basique semble alors mieux que l'aléatoire.
		En effet on se base sur la moyenne des notes ainsi que la moyenne utilisateur et moyenne du film.
		
	\section{Méthode de voisinage}
		La première remarque que nous pouvons faire est que le temps de calcul est grandement augmenté.
	
\end{document}
